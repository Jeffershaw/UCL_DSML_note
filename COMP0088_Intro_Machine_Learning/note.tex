% Use only LaTeX2e, calling the article.cls class and 12-point type.

\documentclass[12pt,a4paper]{article}

% Users of the {thebibliography} environment or BibTeX should use the
% scicite.sty package, downloadable from *Science* at
% www.sciencemag.org/about/authors/prep/TeX_help/ .
% This package should properly format in-text
% reference calls and reference-list numbers.
\usepackage[numbers, sort&compress]{natbib}
%\usepackage[square]{natbib}
%\usepackage{scicite}
\usepackage{graphicx}
\usepackage{amssymb}
\usepackage{geometry}
\usepackage{titlesec}
\usepackage{amsmath}
\usepackage{eqnarray}
% Use times if you have the font installed; otherwise, comment out the
% following line.

\usepackage{times}

% The preamble here sets up a lot of new/revised commands and
% environments.  It's annoying, but please do *not* try to strip these
% out into a separate .sty file (which could lead to the loss of some
% information when we convert the file to other formats).  Instead, keep
% them in the preamble of your main LaTeX source file.


% The following parameters seem to provide a reasonable page setup.
\geometry{a4paper,scale=0.8}
\topmargin 0.0cm
\oddsidemargin 0.2cm
\textwidth 20cm 
\textheight 21cm
\footskip 1.0cm
\geometry{top=1.5cm}
%The next command sets up an environment for the abstract to your paper.
\newenvironment{sciabstract}{%
\begin{quote} \bf}
{\end{quote}}
\renewcommand\refname{References and Notes}
\newcounter{lastnote}
\newenvironment{scilastnote}{%
\setcounter{lastnote}{\value{enumiv}}%
\addtocounter{lastnote}{+1}%
\begin{list}%
{\arabic{lastnote}.}
{\setlength{\leftmargin}{.22in}}
{\setlength{\labelsep}{.5em}}}
{\end{list}}


% Include your paper's title here

\title{Introduction to Machine Learning} 

%\author{Jiafeng Shou} 
\time 0
\begin{document} 
% Double-space the manuscript.

\baselineskip24pt
% Make the title.
\maketitle 
%\tableofcontents

\section{Lecture 1}
% \begin{eqnarray}
% norm({\rm {\bf W}_i(\theta_j)})=\sqrt{x_1^2+x_2^2+\ldots+x_n^2}\\
% {\rm {\bf \bar W}_i(\theta_j)} = \frac{\rm {\bf W}_i(\theta_j)}{norm({\rm {\bf W}_i(\theta_j)})}
% \end{eqnarray}
%$$\int_{2}^{3}x^2dx$$
\subsection*{linear function}
$$
y = f_{\bf W}({\bf X})=f({\bf X,W})={\bf W^{\rm T}X}
$$
\subsubsection*{linear classifier (perception model)}
$$
{\bf x}_i \cdot {\bf w}+b \geq 0
$$
$$
{\bf x}_i \cdot {\bf w}+b < 0
$$
\subsubsection*{linear regression in 1 dimension}
$$
y^i = {\bf W}^{\rm T}{\bf X}^i+\epsilon^i
$$
where $\epsilon$ is the noise(loss).\\
Loss function: sum of squared errors
$$
L({\bf W}) = \sum_{i=1}^{N}(\epsilon^i)
^2$$
$$
L(w_0,w_1) = \sum_{i=1}^{N}
$$
%\nabla
$$
\frac{\partial L(w_0,w_1)}{\partial w_0} = \sum_{i=1}^{N} \frac{\partial[y^i-(w_0x_0^i+w_1x_1^i)]^2}{\partial w_0} = -2\sum_{i=1}^{N}(y^i-(w_0x_0^i+w_1x_1^i))x_0^i = 0
$$
$$
\sum_{i=1}^{N}y^ix^i_0 = w_0\sum_{i=1}^{N}x_0^ix_0^i+w_1\sum_{i=1}^{N}x_1^ix_0^i
$$
as follow, the partial gradient of $w_1$ would be 
$$
\sum_{i=1}^{N}y^ix^i_1 = w_0\sum_{i=1}^{N}x_0^ix_1^i+w_1\sum_{i=1}^{N}x_1^ix_1^i
$$
Therefore
\begin{equation}
\left[
\begin{matrix}
\sum_{i=1}^{N}y^ix_0^i \\
\sum_{i=1}^{N}y^ix_1^i \\
\end{matrix}
\right]=
\left[
\begin{matrix}
\sum_{i=1}^{N}x^i_0x_0^i & \sum_{i=1}^{N}x_0^ix_1^i \\
\sum_{i=1}^{N}x_0^ix_1^i & \sum_{i=1}^{N}x_1^ix_1^i \\
\end{matrix}
\right]
\left[
\begin{matrix}
w_0 \\
w_1 \\
\end{matrix}
\right]
\end{equation}
Formally, it could conclude that
$$
{\bf X}^{\rm T}{\bf y} = {\bf X}^{\rm T}{\bf X}{\bf w}
$$
$$
{\bf w} = ({\bf X}^{\rm T}{\bf X})^{-1}{\bf X}^{\rm T}{\bf y}
$$
\subsubsection*{Why the gradient could be equal to 0}
Hessian matrix is a square matrix of second-order partial derivatives of scalar-valued function, or scalar field.
\begin{equation}
{\bf H}(f) = \left[
\begin{matrix}
\frac{\partial^2f}{\partial x_1^2} & \frac{\partial^2f}{\partial x_1 \partial x_2} & \ldots & \frac{\partial^2f}{\partial x_1 \partial x_n}\\
\frac{\partial^2f}{\partial x_2 \partial x_1} & \frac{\partial^2f}{\partial x_2^2} & \ldots & \frac{\partial^2f}{\partial x_2 \partial x_n}\\
\vdots & \vdots & \ddots & \vdots\\
\frac{\partial^2f}{\partial x_n \partial x_1} & \frac{\partial^2f}{\partial x_n \partial x_2} & \ldots & \frac{\partial^2f}{\partial x_n^2}\\
\end{matrix}
\right]
\end{equation}
$$
{\bf H} = \left|
\begin{matrix}
\frac{\partial^2f}{\partial x^2} & \frac{\partial^2f}{\partial x \partial y}\\
 \frac{\partial^2f}{\partial y \partial x} & \frac{\partial^2f}{\partial x^2}\\
\end{matrix}
\right|
$$
In the 2 dimension, when ${\bf H}>0$: if $\frac{\partial^2f}{\partial x^2}>0$, then point$(x_0,y_0)$ is the local min point. If $\frac{\partial^2f}{\partial x^2}<0$, then point$(x_0,y_0)$ is the local max point.\\
when ${\bf H}<0$, then point$(x_0,y_0)$ is the  stationary point.\\
when ${\bf H}=0$, second order cannot decide the point property, then consider it in higher order Taylor's Expansion.\\
In the example,${\bf H} = 4(x_0^i)^2(x_1^i)^2-4(x_1^ix_0^i)(x_0^ix_1^i)=0$, and $\frac{\partial^2f}{\partial x^2}=4(x_0^i)^2(x_1^i)^2>0$. Therefore, it is a local min point for the loss function.
\\
In higher dimension space (multi-variables), ${\bf H}(f)$ should be a positive definite matrix($(\nabla {\bf x})^{\rm T}{\bf H}(f)\nabla {\bf x}\geq 0$ for any $\nabla {\bf x}$).\\
The more detail of Hessian matrix could look up Taylor expansion.
\subsubsection*{Generalized linear regression}
$$
L({\bf w}) = \sum_{i=1}^{N}(y^i-{\bf w}^{\rm T}\phi({\rm x}^i))^{\rm T}
$$
where $\phi({\bf x}^i)$ is a polynomial function for ${\bf x}^i$
\subsubsection*{normalization}
L1 norm(euclidean) norm:
$$
||{\bf w}||_2 = \sqrt{\sum_{d=1}^{D}w_d^2}=\sqrt{<{\bf w},{\bf w}>}
$$\\
L2 norm(manhattan) norm:
$$
||{\bf w}||_1 = \sum_{d=1}^{D}|w_d|
$$\\
Lp norm, $p>1$:
$$
||{\bf w}||_p =(\sum_{d=1}^{D}w^p_d)^{\frac{1}{p}} 
$$
\subsection*{Ridge regression: L2-regularized linear regression}
\begin{eqnarray*}
L({\bf w}) &=& {\bf \epsilon}^{\rm T}{\bf \epsilon}+\lambda{\bf w}^{\rm T}{\bf w}={\bf y}^{\rm T}{\bf y}-2{\bf y}^{\rm T}{\bf X}{\bf w}+{\bf w}^{\rm T}({\bf X}^{\rm T}{\bf X}+\lambda{\bf I}){\bf w} \\
\nabla L({\bf w}^*) &=& 0 \\
{\bf w}^* &=& ({\bf X}^{\rm T}{\bf X}+\lambda{\bf I})^{-1}{\bf X}^{\rm T}{\bf y}
\end{eqnarray*}
In some case, the matrix cannot be inversed, then add $\lambda {\bf I}$ to make it invertible..\\
invertible matrix
\subsection*{Lasso regression: L1-regularized linear regression}
%\nabla L({\bf w}^*) = 0
suitable for small sample, large dimension.\\
It could shrink some coefficient into 0, helpful for feature selection.\\
The optimization would be gradient descent, LARS, PGD.
\end{document}




















