% Use only LaTeX2e, calling the article.cls class and 12-point type.

\documentclass[12pt,a4paper]{article}

% Users of the {thebibliography} environment or BibTeX should use the
% scicite.sty package, downloadable from *Science* at
% www.sciencemag.org/about/authors/prep/TeX_help/ .
% This package should properly format in-text
% reference calls and reference-list numbers.
\usepackage[numbers, sort&compress]{natbib}
%\usepackage[square]{natbib}
%\usepackage{scicite}
\usepackage{graphicx}
\usepackage{amssymb}
\usepackage{geometry}
\usepackage{titlesec}
\usepackage{amsmath}
% Use times if you have the font installed; otherwise, comment out the
% following line.

\usepackage{times}

% The preamble here sets up a lot of new/revised commands and
% environments.  It's annoying, but please do *not* try to strip these
% out into a separate .sty file (which could lead to the loss of some
% information when we convert the file to other formats).  Instead, keep
% them in the preamble of your main LaTeX source file.


% The following parameters seem to provide a reasonable page setup.
\geometry{a4paper,scale=0.8}
\topmargin 0.0cm
\oddsidemargin 0.2cm
\textwidth 20cm 
\textheight 21cm
\footskip 1.0cm
\geometry{top=1.5cm}
%The next command sets up an environment for the abstract to your paper.
\newenvironment{sciabstract}{%
\begin{quote} \bf}
{\end{quote}}
\renewcommand\refname{References and Notes}
\newcommand\independent{\protect\mathpalette{\protect\independenT}{\perp}}
\def\independenT#1#2{\mathrel{\rlap{$#1#2$}\mkern2mu{#1#2}}}
\newcounter{lastnote}
\newenvironment{scilastnote}{%
\setcounter{lastnote}{\value{enumiv}}%
\addtocounter{lastnote}{+1}%
\begin{list}%
{\arabic{lastnote}.}
{\setlength{\leftmargin}{.22in}}
{\setlength{\labelsep}{.5em}}}
{\end{list}}


% Include your paper's title here

\title{Introduction to Statistical Data Science} 

%\author{Jiafeng Shou} 
\time 0
\begin{document} 
% Double-space the manuscript.

\baselineskip24pt
% Make the title.
\maketitle 
%\tableofcontents

\section{Lecture 1}
\subsection*{Basic Rules of Probability}
Frequently we will say $p(x) \propto f(x)$ for some non-negative function $f(x)$\\
Then we can conclude that:
$$
p(x) = \frac{f(x)}{\sum_{y}f(y)}
$$
For joint distribution,
$$
\sum_{x}p(x) = \sum_{x}\sum_{y}p(x,y)=1
$$
\subsubsection*{Independence}
If $p(x|y)=p(x)$ for all states of $x$ and $y$, then the variables $x$ and $y$ are said to be independent as $x \independent y$.\\
If $x$ and $y$ are independent, then $x$ and $y$ are uncorrelated. However, in general, $x$ and $y$ are uncorrelated, then cannot conclude that $x$ and $y$ are independent.\\
\subsubsection*{Conditional Independence}
$$
{\mathcal X} \independent {\mathcal Y}|{\mathcal Z}
$$
$$
p({\mathcal X},{\mathcal Y}|{\mathcal Z}) = p({\mathcal X} |{\mathcal Z})p({\mathcal Y}|{\mathcal Z})
$$ 
and
$$
p({\mathcal X}|{\mathcal Y},{\mathcal Z})=p({\mathcal X}|{\mathcal Z})
$$
Conditional independence does not imply marginal independence:
$$
p(x,y) = \sum_{z}p(x|z)p(y|z)p(z) \neq \sum_{z}p(x|z)p(z) \sum_{z}p(y|z)p(z)
$$
\section{Lecture 2}
\subsection*{Graphs}
{\large Definition:}\\
A graph consists of nodes (vertixes) and undirected or directed links (edges) between nodes.\\
{\large Path:}\\
A path from $X_i$ to $X_j$ is a sequence of connected nodes starting at $X_i$ and ending at $X_j$. (no direction)\\
{\large Directed Acyclic Graph:}\\
Graph in which by following the direction of the arrows a node will \textbf{never} be visited \textbf{more than once}.\\
{\large Parents and Children:}\\
Xi is a parent of $X_j$ if there is a link from $X_i$ to $X_j$. Xi is a child of $X_j$ if there is a link from $X_j$ to $X_i$.\\
{\large Ancestors and Descendants:}\\
The ancestors of a node $X_i$ are the nodes with a directed path ending at $X_i$. The descendants of $X_i$ are the nodes with a directed path beginning at $X_i$.\\
\subsection*{Undirected Graph:}
{\large Clique:}\\
A clique is a fully connected subset of nodes.\\
{\large Maximal Clique:}\\
Clique which is not a subset of a larger clique.\\
{\large Connected graph:}\\
There is a path between every pair of vertices.\\
{\large Connected components:}\\
In a non-connected graph, the connected components are the connected-subgraphs.\\
{\large Connectedness: Singly-connected}\\
There is only one path from any node $\alpha$ to another node $b$\\
{\large Multiply-connected}\\
A graph is multiply-connected if it is not singly connected.\\

\subsection*{Belief Networks (Bayesian Networks)}
A belief network is a \textbf{directed acyclic graph} in which each node is associated with the conditional probability of the node given its parents.
\subsubsection*{Processing the network}
Firstly write the whole joint distribution such as:
$$
p(A,R,E,B) = p(A|R,E,B)p(R|E,B)p(E|B)p(B)
$$
Then, according to the assumption, remove some independent variable from the joint distribution. \textbf{It does matter that the order of joint distribution influence the processing}.
\subsubsection*{Uncertain Evidence}
In soft/uncertain evidence the variable is in more than one state, with the strength of our belief about each state being given by probabilities. For example, if $y$ has the states $dom(y) =$ \{red, blue, green\} the vector $(0.6, 0.1, 0.3)$ could represent the probabilities of the respective states\\
In the calculation, we can do this: Given $P(A=tr)=0.7$ 
\begin{eqnarray*}
p(B=tr|\widetilde{A}) = \sum_{A}p(B=tr|A)p(A|\widetilde{A}) 
\end{eqnarray*}
\subsubsection*{Independence}
If $C$ has more than one incoming link, then $A\independent B$ and A is not conditional independent with B under C condition. In this case C is called collider.
If C has at most one incoming link, then $A \independent B|C$ and A is not independent with B. In this case $C$ is called non-collider.
\subsubsection*{d-connected/separated}
\subsubsection*{Markov Equivalence}
\subsubsection*{BN representation}
\end{document}




















