% Use only LaTeX2e, calling the article.cls class and 12-point type.

\documentclass[12pt,a4paper]{article}

% Users of the {thebibliography} environment or BibTeX should use the
% scicite.sty package, downloadable from *Science* at
% www.sciencemag.org/about/authors/prep/TeX_help/ .
% This package should properly format in-text
% reference calls and reference-list numbers.
\usepackage[numbers, sort&compress]{natbib}
%\usepackage[square]{natbib}
%\usepackage{scicite}
\usepackage{graphicx}
\usepackage{amssymb}
\usepackage{geometry}
\usepackage{titlesec}
\usepackage{amsmath}
% Use times if you have the font installed; otherwise, comment out the
% following line.

\usepackage{times}

% The preamble here sets up a lot of new/revised commands and
% environments.  It's annoying, but please do *not* try to strip these
% out into a separate .sty file (which could lead to the loss of some
% information when we convert the file to other formats).  Instead, keep
% them in the preamble of your main LaTeX source file.


% The following parameters seem to provide a reasonable page setup.
\geometry{a4paper,scale=0.8}
\topmargin 0.0cm
\oddsidemargin 0.2cm
\textwidth 20cm 
\textheight 21cm
\footskip 1.0cm
\geometry{top=1.5cm}
%The next command sets up an environment for the abstract to your paper.
\newenvironment{sciabstract}{%
\begin{quote} \bf}
{\end{quote}}
\renewcommand\refname{References and Notes}
\newcounter{lastnote}
\newenvironment{scilastnote}{%
\setcounter{lastnote}{\value{enumiv}}%
\addtocounter{lastnote}{+1}%
\begin{list}%
{\arabic{lastnote}.}
{\setlength{\leftmargin}{.22in}}
{\setlength{\labelsep}{.5em}}}
{\end{list}}


% Include your paper's title here

\title{Introduction to Statistical Data Science} 

%\author{Jiafeng Shou} 
\time 0
\begin{document} 
% Double-space the manuscript.

\baselineskip24pt
% Make the title.
\maketitle 
%\tableofcontents

\section{Lecture 1}
% \begin{eqnarray}
% norm({\rm {\bf W}_i(\theta_j)})=\sqrt{x_1^2+x_2^2+\ldots+x_n^2}\\
% {\rm {\bf \bar W}_i(\theta_j)} = \frac{\rm {\bf W}_i(\theta_j)}{norm({\rm {\bf W}_i(\theta_j)})}
% \end{eqnarray}
%$$\int_{2}^{3}x^2dx$$

\subsection{Normal Distribution}
$$
p(x) = \frac{1}{\sqrt{2\pi\sigma^2}}exp{(-\frac{(x-u)^2}{2\sigma^2})}
$$

$$
F(x)\equiv P(X\leq x)
$$
\\
{\bf{Central Limit Theorem (for normal distribution)}}\\
The more random variables we average over, the closer the resulting
distribution will be to the Normal distribution
\\
{\bf Parameter: $u$}
The mean is the location parameter.\\
{\bf Parameter: $\sigma^2$}
The variance is the scale parameter\\

\subsection{Uniform Distribution}
\begin{eqnarray}
X \sim U[0,1] \\
0\leq x \leq 1\\
p(x) = 1 \\
F(x) = P(X\leq X)=x\\
\end{eqnarray}
{\bf Use uniform distribution to construct normal distribution}\\
\begin{equation}
{\bf X} = \left[
\begin{array}{c}
X^{(1)} \\
X^{(2)} \\
\ldots \\
X^{(n)} \\
\end{array}
\right]
\end{equation}
$$
X^{(i)} \sim U[0,1]
$$
$$
Y = \frac{1}{n}\sum_{i=1}^{n}X^{(i)}
$$\\
subsampling $\bf X$ vector to construct $Y$. The distribution of $Y_j \sim ?, j=1,2,3,\ldots,p$ would close to normal distribution.

\subsection{Poisson distribution}
$$
P(X=k) = \frac{e^{-\lambda}\lambda^k}{k!}
$$
$$
E(X)=V(X)=\lambda
$$

\subsection{empirical CDF}
$$
F_n(x)=\frac{1}{n}\sum_{i=1}^{n}I(x^{(i)}\leq x)
$$
quantile is simply the inverse of the CDF:
	-The 0.9 quantile is the value of $x$ such that $F(x)=0.9$ i.e. $x=F^{-1}(0.9)$
\section{Hypothesis}
\subsection*{null hypothesis}
In inferential statistics, the null hypothesis is a general statement or default position that there is no relationship between two measured phenomena, or no association among groups.
\subsection*{p-value}
The probability of obtaining results as or more extreme than that observed, assuming $H_0$ is true, is the p-value, under the assumption that the null hypothesis.
$$
p\equiv P(X\leq 15;H_0)
$$
$$
p=\sum_{x=0}^{15}\bigl(\begin{smallmatrix}40\\x\end{smallmatrix}\bigr)0.5^x(1-0.5)^{(40-x)}
$$
The p-value is most certainly not the probability of $H_0$ being true \\
When used in practice with a threshold of 0.05 this is an informal
method of reasoning and can be easily criticized\\
\subsection*{Statistical power}
The power of a hypothesis test is the probability of avoiding a false
negative
\subsection*{P-value distribution (CDF)}
$$
P(F(X)\leq z) = P(F^{-1}(F(x))\leq F^{-1}(z)) = P(X\leq F^{-1}(z))=F(F^{-1}(z))=z
$$
\subsubsection*{Type 1 errors}
\subsubsection*{Type 2 errors}
\subsubsection*{power of the test}
\subsubsection*{critical region}
\subsubsection*{Composite hypothesis}

\end{document}




















